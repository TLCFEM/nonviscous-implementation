\documentclass[3p,11pt,fleqn,review,sort&compress]{elsarticle}
\bibliographystyle{elsarticle-num-names}
\usepackage{amsmath,amsfonts,amssymb,siunitx,float,subcaption,setspace,booktabs,diagbox,bm,tikz,structmech,mathtools,tikz-3dplot,pgfplots,gnuplot-lua-tikz,url,fontawesome5}
\usepackage{algorithm}
\usepackage{algpseudocode}
\usepackage{lineno}
\usepackage[many]{tcolorbox}
\usetikzlibrary{shapes.geometric}
\pgfplotsset{compat=1.8}
\journal{}
\newcommand*{\mb}[1]{\bm{#1}}
\newcommand*{\bbar}[1]{\bar{\bm{#1}}}
\newcommand*{\bhat}[1]{\hat{\bm{#1}}}
\newcommand*{\mT}{\mathrm{T}}
\newcommand*{\md}[1]{\mathrm{d}#1}
\newcommand*{\eqsref}[1]{Eq.~(\ref{#1})}
\newcommand*{\figref}[1]{Fig.~\ref{#1}}
\newcommand*{\tabref}[1]{Table~\ref{#1}}
\newcommand*{\algoref}[1]{Algorithm~\ref{#1}}
\newcommand*{\secref}[1]{\S~\ref{#1}}
\newcommand*{\diag}[1]{\text{diag}\left(#1\right)}
\newcommand*{\sign}[1]{\text{sign}\left(#1\right)}
\newcommand*{\dev}[1]{\text{dev}\left(#1\right)}
\newcommand*{\tr}[1]{\text{trace}\left(#1\right)}
\newcommand*{\ddfrac}[2]{\dfrac{\md{#1}}{\md{#2}}}
\newcommand*{\pdfrac}[2]{\dfrac{\partial{#1}}{\partial{#2}}}
\newcommand{\bsigma}{\mb{\sigma}}
\newcommand{\bvarepsilon}{\mb{\varepsilon}}
\newcommand{\beeta}{\mb{\eta}}
\newcommand{\bn}{\mb{n}}
\newcommand{\balpha}{\mb{\alpha}}
\newcommand{\bbeta}{\mb{\beta}}
\newcommand{\bgamma}{\mb{\gamma}}
\newcommand{\bq}{\mb{q}}
\newcommand{\bs}{\mb{s}}
\newcommand{\bc}{\mb{c}}
\newcommand{\be}{\mb{e}}
\newcommand{\bu}{\mb{u}}
\newcommand{\bv}{\mb{v}}
\newcommand{\bw}{\mb{w}}
\newcommand{\ba}{\mb{a}}
\newcommand{\bx}{\mb{x}}
\newcommand{\bbf}{\mb{f}}
\def\thefootnote{\fnsymbol{footnote}}
\DeclarePairedDelimiter\abs{\lvert}{\rvert}
\DeclarePairedDelimiter\norm{\lVert}{\rVert}
\tcbset{
    sharp corners,
    colback = white,
    before skip = 0.2cm,    % add extra space before the box
    after skip = 0.5cm      % add extra space after the box
}                           % setting global options for tcolorbox
\newtcolorbox{Objective}{
    sharpish corners, % better drop shadow
    boxrule = 0pt,
    leftrule = 4.5pt, % top rule weight
    enhanced,
    fuzzy shadow = {0pt}{-2pt}{-0.5pt}{0.5pt}{black!35} % {xshift}{yshift}{offset}{step}{options}
}
\makeatletter
\let\oldabs\abs
\def\abs{\@ifstar{\oldabs}{\oldabs*}}
\let\oldnorm\norm
\def\norm{\@ifstar{\oldnorm}{\oldnorm*}}
\newenvironment{breakablealgorithm}
{\begin{center}
\refstepcounter{algorithm}
\hrule height.8pt depth0pt \kern2pt
\renewcommand{\caption}[2][\relax]{
{\raggedright\textbf{\ALG@name~\thealgorithm} ##2\par}
\ifx\relax##1\relax
\addcontentsline{loa}{algorithm}{\protect\numberline{\thealgorithm}##2}
\else
\addcontentsline{loa}{algorithm}{\protect\numberline{\thealgorithm}##1}
\fi
\kern2pt\hrule\kern2pt
}}{\kern2pt\hrule\relax
\end{center}}
\begin{document}
\linenumbers
\begin{abstract}
\begin{linenumbers}
In this work, we propose a fast algorithm to evaluate the dynamic response of nonviscously damped systems with arbitrary distributions of damping characterised by arbitrary kernel functions (both number and form). The proposed algorithm is independent from time integration methods and requires mere level 2 BLAS operations.
\end{linenumbers}
\end{abstract}
\begin{keyword}
\end{keyword}
\begin{frontmatter}
\title{A Strategy for Fast Evaluation of Nonviscously Damped Systems With Arbitrary Kernels}
%\author[add1]{Theodore~L.~Chang\corref{tlc}}\ead{tlcfem@gmail.com}
%\author[add2]{Chin-Long~Lee}
%\cortext[tlc]{corresponding author}
%\address[add1]{IRIS Adlershof, Humboldt-Universität zu Berlin, Berlin, Germany, 12489.}
%\address[add2]{Department of Civil and Natural Resources Engineering, University of Canterbury, Christchurch, New Zealand, 8041.}
\end{frontmatter}
\section{Introduction}
The equation of motion of nonviscously damped systems can be, conventionally, expressed as an integro-differential equation, namely, for elastic linear single-degree-of-freedom systems,
\begin{gather}\label{eq:single_eom}
m\ddot{u}\left(t\right)+\left(g*\dot{u}\right)\left(t\right)+ku\left(t\right)=p\left(t\right),
\end{gather}
where $u(t)$ denotes the displacement, $\dot{(\cdot)}$ denotes time derivative, and $g=g(t)$ is the kernel function. Various forms have been proposed, see a summary by \citet[][Table 1]{Adhikari2003}.
The convolution term in \eqsref{eq:single_eom} can be expressed in integral form such that
\begin{gather}\label{eq:conv}
\left(g*\dot{u}\right)\left(t\right)=\int_0^tg\left(t-\tau\right)\dot{u}\left(\tau\right)\md{\tau}.
\end{gather}
To solve \eqsref{eq:single_eom} numerically, \eqsref{eq:conv} needs to be evaluated. Different approaches are available, we discuss major ones in the following.
\subsection{Direct/Explicit Integration Methods}
Assuming $\dot{u}(t_i)$ is already obtained up to a given time $T$ at each sampling point $t_i$, \eqsref{eq:conv} can be numerically integrated such that
\begin{gather}
\int_0^Tg\left(T-\tau\right)\dot{u}\left(\tau\right)\md{\tau}\approx\sum_i^n\omega_i\dot{u}\left(t_i\right),
\end{gather}
where $t_i=\left\{t_0,t_1,t_2,\cdots,t_n\right\}\in[0,T]$ that can be either evenly or unevenly spaced, and $\omega_i$ is the corresponding integration weight that may directly contain or can be associated \citep{Schaedle2006} with the term $g\left(T-t_i\right)$. If the exact $\dot{u}(t_i)$ is not available, approximations (interpolation, weighted sum, etc.) based on adjacent known values can be adopted. A generalisation adopts fractional calculus \citep[e.g.,][]{Bagley1983,Gaul1999} in $g(t)$, which yields a simple form in the frequency domain, relevant analytical techniques can be applied. For further discussions, see, for example, the work by \citet{Fernandez2019}.

Such a direct approach, known as convolution quadrature \citep[see][and the references therein]{Lubich2004}, is general--purpose such that it can be used for both linear and nonlinear systems with either explicit or implicit time integration methods \citep[see, e.g.,][]{Katsikadelis2019}. The simplest integration one can come up with is the rectangle rule, which is used in the implementation by \citet{Puthanpurayil2014}. Noting that the rectangle rule possesses the lowest order of accuracy possible, it effectively yields low-order overall accuracy --- even when the adopted time integration method is second-order accurate. To improve, trapezoidal rule \citep[e.g.,][]{Liu2014} and Simpson's rules \citep[e.g.,][]{Shen2019} can be used. If necessary, other high-order Newton--Cotes rules and/or more complex approximations \citep{Schaedle2006,Shen2021} can also be adopted.

However, no matter how the convolution term is integrated, it has to be performed for \textbf{each} substep and the whole velocity history needs to be stored. Given the nature of convolution, most intermediate results computed in the current substep cannot be reused in subsequent substeps. The potential exception is the values of $g$ evaluated at $T-t_i$, which do not need to be re-evaluated if $T$ and $t_i$ satisfy certain conditions. For response history analyses, since velocity is computed at discrete time moments, existing methods typically use a fixed time step size and the same set of $t_i$ in both response history analysis (to define time substeps) and numerical integration of \eqsref{eq:conv} (as the abscissae), this gives a complexity of $\mathcal{O}\left(n^2\right)$ where $n$ denotes the number of substeps. The explicit integration also requires additional kinetic assumptions, that may conflict against the ones adopted by the corresponding time integration method, to be imposed. They have significant impacts on the final analysis result \citep[see][]{Liu2014} and may even lead to poor result \cite[see][Figs. 12, 17, 25, 26]{Liu2023}.

In the context of dynamic analysis, since most direct time integration methods (such as the Newmark method) are second-order accurate, \eqsref{eq:conv} has to be integrated with methods with at least a second-order accuracy. For methods of this kind, a potential improvement of overall computational efficiency is to use a fast converging method \citep{Schaedle2006} such that a large step size can be used for convolution quadrature. As a result, fewer velocity snapshots need to be stored, and the constant factor in complexity $\mathcal{O}\left(n^2\right)$ can be lowered.

Alternatively, noting that $\mathcal{L}\left(f*g\right)=\mathcal{L}\left(f\right)\cdot\mathcal{L}\left(g\right)$ where $\mathcal{L}$ stands for Laplace transform, it is possible to compute \eqsref{eq:conv} in the frequency domain via fast Fourier transform, rather than directly in the time domain, relevant applications can be seen elsewhere \citep{Pan2013,Zhao2019}.
\subsection{State Space Methods}
With the assist of Laplace transform, the convolution term involving (sum of) exponential kernels (and potentially other functions) can be eliminated at the cost of increasing the order of time derivatives \citep[see, e.g.,][]{Wu2019}. By further introducing additional state variables, the second-order system \eqsref{eq:single_eom} can be converted into a first-order one, with which methods such as modal analysis\footnote{Noting that polynomial eigenvalue problems are essentially equivalent to generalised eigenvalue problems.} can be applied to obtain analytical solutions for linear systems.

Although state space methods allow analysts to obtain closed-form solutions, only kernels that possess simple forms in $s$-domain can be considered. If the characteristic equation in $s$-domain is a transcendental function or a higher order polynomial (the Abel--Ruffini theorem), the system would still be solved numerically in the sense that the eigenvalues can only be computed imprecisely. The converted first-order ODE can also be solved via numerical integration \citep{Adhikari2004}. However, since the order is lowered, its size increases accordingly. This often drastically increases computational costs. For further discussions on methods of this kind, one can refer to the summary by \citet[][\S~1.3.1]{Adhikari2014}.
\subsection{Need of An Efficient Approach}
Based on the above discussion, nonviscously damped systems would better be solved by an algorithm that possesses the following attributes.
\begin{Objective}
\begin{enumerate}
\item It preserves the size of the original dynamic system.
\item It maintains the accuracy of the time integration method used.
\item It eliminates the need to integrate the convolution term for each substep.
\item It can be used with various time integration methods and kernels in both elastic/inelastic systems.
\end{enumerate}
\end{Objective}

This work presents an algorithm to incorporate arbitrary smooth kernels with direct time integration methods and explains the strategy in two parts.
\begin{enumerate}
\item[\secref{sec:single}] For exponential kernels or kernels that can be expressed as sums of exponentials, a nonviscously damped system can be equivalently rewritten into a second-order ODE and a first-order ODE. Both can be discretised and solved directly in the time domain with a performant formulation.
\item[\secref{sec:arbitrary}] Arbitrary smooth kernels can be approximated by sums of exponentials in a way that is independent of the dynamic system of interest. The approximation balances well between accuracy and efficiency.
\end{enumerate}
Examples are presented with an emphasis on accuracy by comparing numerical results with analytical solutions (if available).

\section{An Efficient Algorithm for Systems With A Single Exponential Kernel}\label{sec:single}
The present section introduces an enhancement of the algorithm proposed by \citet{Adhikari2004} through optimisation. The improvement is accomplished by reformulating the algorithm into a form that can be extended to account for arbitrary kernel functions.
\subsection{Nonviscously Damped MDOF System}
Without loss of generality, instead of \eqsref{eq:single_eom}, consider the equation of motion of a nonviscously damped \textit{inelastic} multi-degree-of-freedom system,
\begin{gather}\label{eq:eom}
\mb{y}+\bbf=\mb{p},
\end{gather}
where $\mb{y}=\mb{y}\left(\bu,\bv,\ba\right)$ is the resistance vector of the system, $\mb{p}=\mb{p}\left(t\right)$ is the external load vector as in \eqsref{eq:single_eom}, and $\bbf$ is the nonviscous damping force which can be expressed in the form of the convolution of the kernel $g=g\left(t\right)$ and the vector $\bw$, viz., $\bbf\left(t\right)=\left(g*\bw\right)\left(t\right)$.

Note here, $\bw$ can be either the exact velocity vector $\bv$, or a vector that depends on $\bv$ (e.g., a subset of $\bv$). Formally, it can be represented by
\begin{gather}\label{eq:wv}
\bw=\mb{T}\bv,
\end{gather}
where $\mb{T}$ picks the participating DoFs, and can be, for example, a square diagonal matrix with its diagonal entries being either one or zero. If $\mb{T}$ is the identity matrix, then $\bw=\bv$, the convolution is expressed in the conventional form. \eqsref{eq:wv} is beneficial when it comes to compositing flexible damping based on different node/element-based rules. It will be discussed later in this work.
\subsection{A Single Exponential Kernel}
We start with the scalar-valued exponential kernel function
\begin{gather}
g=g\left(t\right)=m\exp\left(-st\right),
\end{gather}
where $s$ is often denoted by the relaxation parameter $\mu$, $m$ is often denoted by $c\mu$ in which $c$ is the damping constant. In this work, $s$ and $m$ are adopted for brevity.
The convolution can be then expressed as
\begin{gather}\label{eq:single_conv}
\bbf\left(t\right)=\left(g*\bw\right)\left(t\right)=\int_0^tg(t-\tau)\cdot\bw\left(\tau\right)~\md{\tau}=\int_0^tm\exp\left(-s\left(t-\tau\right)\right)\cdot\bw\left(\tau\right)~\md{\tau}.
\end{gather}
\eqsref{eq:single_conv} corresponds to the solution of the following ODE \citep[see, e.g.,][\S~80]{Zwillinger2021},
\begin{gather}\label{eq:single_conv_ode}
\dot{\bbf}=-s\bbf+m\bw.
\end{gather}
It can be validated by solving \eqsref{eq:single_conv_ode} with the assistance of the integrating factor $\exp\left(st\right)$. Here it is assumed that $\bbf\left(0\right)=\mb{0}$. \eqsref{eq:single_conv_ode} and its similar forms are widely used in constitutive modelling in terms of viscoelastic materials \citep[e.g.,][]{Muravyov1998} and hardening rules \citep[e.g.,][]{Chaboche1989}.
\subsection{An Efficient Direct Time Integration Algorithm}
Instead of directly integrating \eqsref{eq:single_conv} using higher-order methods (such as the Runge--Kutta family), to develop an efficient algorithm, \eqsref{eq:single_conv_ode} can be combined with \eqsref{eq:eom} such that the system becomes
\begin{gather}\label{eq:eqv_sys}
\left\{
\begin{array}{l}
\dot{\bbf}=-s\bbf+m\bw,\\
\mb{y}+\bbf=\mb{p},
\end{array}\right.
\end{gather}
in which the first equation is a first-order ODE (of $\bbf$) while the second equation is a second-order ODE (of $\bu$).

Given that popular time integration methods are of second-order accuracy, in the context of a discretised iterative solving schema, \eqsref{eq:single_conv_ode} can be rewritten as follows using the (implicit) trapezoidal rule, which is second-order accurate and A-stable,
\begin{gather}
\bbf_{n+1}=\bbf_n+\dfrac{\Delta{}t}{2}\left(\dot{\bbf}_n+\dot{\bbf}_{n+1}\right).
\end{gather}
Expanding gives
\begin{gather}
\bbf_{n+1}=\bbf_n+\dfrac{\Delta{}t}{2}\left(\left(-s\bbf_{n}+m\bw_{n}\right)+\left(-s\bbf_{n+1}+m\bw_{n+1}\right)\right),
\end{gather}
one may further rearrange to obtain
\begin{gather}\label{eq:discretised_c}
\bbf_{n+1}=\dfrac{2-s\Delta{}t}{2+s\Delta{}t}\bbf_n+\dfrac{m\Delta{}t}{2+s\Delta{}t}\bw_n+\dfrac{m\Delta{}t}{2+s\Delta{}t}\bw_{n+1},
\end{gather}
in which subscripts $\left(\cdot\right)_n$ and $\left(\cdot\right)_{n+1}$ denote the corresponding quantity at $t_n$ and $t_{n+1}=t_n+\Delta{}t$.
It shall be noted that the \textit{implicit} trapezoidal rule is used, however, due to the special form of \eqsref{eq:single_conv_ode}, $\bbf_{n+1}$ can be \textit{explicitly} expressed as in \eqsref{eq:discretised_c}. This allows the following formulation.

Assuming the equation of motion \eqsref{eq:eom} is satisfied at $t_{n+1}$, then, accounting for \eqsref{eq:discretised_c}, \eqsref{eq:eom} is
\begin{gather}\label{eq:residual}
\mb{y}_{n+1}+\dfrac{2-s\Delta{}t}{2+s\Delta{}t}\bbf_n+\dfrac{m\Delta{}t}{2+s\Delta{}t}\bw_n+\dfrac{m\Delta{}t}{2+s\Delta{}t}\bw_{n+1}=\mb{p}_{n+1}.
\end{gather}
Differentiation leads to the following revised effective stiffness $\bhat{K}_{n+1}$,
\begin{gather}\label{eq:revised_k}
\bhat{K}_{n+1}=\bbar{K}_{n+1}+\dfrac{m\Delta{}t}{2+s\Delta{}t}\mb{T}\ddfrac{\bv_{n+1}}{\bu_{n+1}},
\end{gather}
in which $\bbar{K}_{n+1}$ denotes the conventional effective stiffness that can be expressed as
\begin{gather}
\bbar{K}_{n+1}=\ddfrac{\mb{y}_{n+1}}{\bu_{n+1}}=\mb{K}_{n+1}+\mb{C}_{n+1}\ddfrac{\bv_{n+1}}{\bu_{n+1}}+\mb{M}_{n+1}\ddfrac{\ba_{n+1}}{\bu_{n+1}}.
\end{gather}
The viscous damping matrix $\mb{C}$ is included here for generality. It may not be trivial as the system may consist of viscous damping components (e.g., viscous damper devices). Typically, quantities $\ddfrac{\bv_{n+1}}{\bu_{n+1}}$ and $\ddfrac{\ba_{n+1}}{\bu_{n+1}}$ reduce to scalar constants (multiplied by an identity matrix), for example, in the Newmark method, $\ddfrac{\bv_{n+1}}{\bu_{n+1}}=\dfrac{\gamma}{\beta\Delta{}t}$. It can be seen from \eqsref{eq:revised_k} that to account for nonviscous damping in existing viscous damping algorithms, only the damping matrix needs to be revised such that
\begin{gather}
\bhat{C}_{n+1}=\mb{C}_{n+1}+\dfrac{m\Delta{}t}{2+s\Delta{}t}\mb{T}.
\end{gather}
The extra term $\dfrac{m\Delta{}t}{2+s\Delta{}t}\mb{T}$ involves only scalar-matrix product, and $\mb{T}$ is often sparse or even diagonal (using a node-based distribution rule). It does not alter any other parts of the established solving algorithm (for viscous systems).

By introducing the revised resistance $\bhat{y}_{n+1}$ as
\begin{gather}\label{eq:revised_y}
\bhat{y}_{n+1}=\mb{y}_{n+1}+\dfrac{2-s\Delta{}t}{2+s\Delta{}t}\bbf_n+\dfrac{m\Delta{}t}{2+s\Delta{}t}\bw_n+\dfrac{m\Delta{}t}{2+s\Delta{}t}\bw_{n+1},
\end{gather}
the linearised system to be solved is
\begin{gather}\label{eq:revised_system}
\bhat{K}_{n+1}\delta\bu=\mb{p}_{n+1}-\bhat{y}_{n+1}.
\end{gather}
\begin{Objective}
%\begin{rmk}
The implicit trapezoidal rule is second-order accurate. It is adopted to improve the overall accuracy accounting for the fact that most time integration methods are second-order accurate. Since the implicit trapezoidal rule is A-stable, the overall result is stable as long as the adopted time integration method is stable.

\eqsref{eq:discretised_c} is expressed in a form that is compatible with single-step algorithms. It thus can be used with other single-step time integration methods. Nevertheless, apart from the implicit trapezoidal rule, other methods can be applied to discretise \eqsref{eq:single_conv_ode}. The following are some examples.
\begin{enumerate}
\item For the Bathe two-step method \citep{Noh2019}, the TR-BDF2 method \citep{Bank1985} can be used.
\item For the OALTS method \citep{Zhang2021}, the BDF2 method can be used.
\item For the generalised-$\alpha$ method \citep{Chung1993} and the GSSSS method \citep{Zhou2003}, the variable step size BDF2 method can be used.
\end{enumerate}
%\end{rmk}
\end{Objective}

Given that the discretisation \eqsref{eq:discretised_c} does not introduce additional assumptions regarding the system itself, \eqsref{eq:revised_system} is universal and can be applied to both elastic and inelastic systems. In the context of inelastic systems, the local iteration body is summarised in \algoref{algo:single_model} with subscript $\left(\cdot\right)_{n+1}$ dropped for brevity.
The steps with a leading \faMicrochip~symbol augment global effective stiffness and resistance to obtain $\bhat{K}$ and $\bhat{y}$. Those are additional steps that need to be computed compared to a conventional algorithm for viscously damped systems. Once convergence is achieved, it is necessary to store the history of nonviscous damping force $\bbf_n\leftarrow\bbf$.
\begin{breakablealgorithm}
\setstretch{1.6}
\caption{iteration body of solving nonviscously damped system with one exponential kernel}\label{algo:single_model}
\begin{algorithmic}
\State \textbf{*Schema: Newmark + Trapezoidal}
\State \textbf{Input}: $\bbar{K}$, $\bu$, $\bv$, $\mb{y}$, $\mb{p}$ (quantities obtained via conventional manner as if there is no nonviscous damping) and $\bbf_n$, $m$, $s$
\State \textbf{Output}: $\bu$
\State compute $\bw$ from $\bv$
\State \faMicrochip~compute nonviscous damping force $\bbf=\dfrac{2-s\Delta{}t}{2+s\Delta{}t}\bbf_n+\dfrac{m\Delta{}t}{2+s\Delta{}t}\bw_n+\dfrac{m\Delta{}t}{2+s\Delta{}t}\bw$\Comment{\eqsref{eq:discretised_c}}
\State \faMicrochip~compute revised stiffness $\bhat{K}=\bbar{K}+\dfrac{m\Delta{}t}{2+s\Delta{}t}\mb{T}\ddfrac{\bv}{\bu}$\Comment{\eqsref{eq:revised_k}}
\State \faMicrochip~compute revised resistance $\bhat{y}=\mb{y}+\bbf$\Comment{\eqsref{eq:revised_y}}
\State $\delta\bu=\bhat{K}^{-1}\left(\mb{p}-\bhat{y}\right)$\Comment{\eqsref{eq:revised_system}}
\State update and return $\bu\leftarrow\bu+\delta\bu$
\end{algorithmic}
\end{breakablealgorithm}

Upon careful comparison, it becomes evident that the algorithm presented bears resemblance to the one discussed by \citet{Adhikari2004}. However, there are differences between them. Specifically, \algoref{algo:single_model} tracks the history of nonviscous damping force, and it does not differentiate between full-rank and rank-deficient cases. As a result, it eliminates the need for any supplementary matrix factorisations. \algoref{algo:single_model} directly discretises the EOM \eqsref{eq:eqv_sys} without converting it to a first-order system via the state space.

Unlike other algorithms, such as the one by \citet{Cortes2009}, \algoref{algo:single_model} does not impose additional requirements on the time integration method used. Some existing state space methods manage to eliminate the convolution integral term from the \textbf{continuous} version of \eqsref{eq:eom} \citep[see][]{Wagner2003,Wu2019}, which, in the writers' opinion, overcomplicates the solution procedure in the context of \textit{numerical} analysis, given that the analytical first-order ODE in the state space would still need to be discretised \citep{Adhikari2004} and numerically integrated for general systems. Nonetheless, whenever analytical solutions are sought, those methods may provide extra merits that direct time integration methods do not offer.
\subsection{Complexity Analysis}
If $\mb{T}$ is a diagonal matrix, implying a node-based damping distribution, \eqsref{eq:revised_k} requires $n$ additional floating point number arithmetic while \eqsref{eq:revised_y} requires $3n$, with $n$ denoting the size of the system. The total number of additional floating point number multiplications is $4n$, that is a time complexity of $\mathcal{O}\left(n\right)$. If an element-based damping distribution rule is used, $\mb{T}$ would have a structure similar to that of $\mb{K}$. Assuming $\mb{T}$ contains $m$ nonzero scalars, the total number of additional floating point number multiplications is $3n+m$.

\algoref{algo:single_model} requires no memory reallocation, the additional storage needed is for the nonviscous damping forces $\bbf_n$, implying a space complexity of $n$. The matrix $\mb{T}$ may also need to be stored depending on the specific form it has.
%\subsection{Stability Analysis}
%For brevity, we consider the decoupled single-degree-of-freedom version of \eqsref{eq:eqv_sys} with the Newmark method and the trapezoidal rule \eqsref{eq:discretised_c}. Furthermore, let $\bw=\bv$. To obtain the amplification matrix, one shall compute the first-order form of Newmark's operator first. By premultiplying the approximation formulas by mass $M$ and accounting for the equilibrium at $t_n$ and $t_{n+1}$, the following expressions can be obtained,
%\begin{gather}
%Mv_{n+1}=Mv_n+\Delta{}t\left(1-\gamma\right)Ma_n+\Delta{}t\gamma{}Ma_{n+1},\\
%Mu_{n+1}=Mu_n+\Delta{}t{}Mv_n+\Delta{}t^2\left(\dfrac{1}{2}-\beta\right)Ma_n+\Delta{}t^2\beta{}Ma_{n+1},
%\end{gather}
%with
%\begin{gather}
%Ma_n=p_n-f_n-Cv_n-Ku_n,\\
%Ma_{n+1}=p_{n+1}-f_{n+1}-Cv_{n+1}-Ku_{n+1},
%\end{gather}
%and
%\begin{gather}
%f_{n+1}=\dfrac{2-s\Delta{}t}{2+s\Delta{}t}f_n+\dfrac{m\Delta{}t}{2+s\Delta{}t}v_n+\dfrac{m\Delta{}t}{2+s\Delta{}t}v_{n+1}.
%\end{gather}
%Choose $\mb{x}=\begin{bmatrix}
%f&u&v
%\end{bmatrix}^\mT$ as the state variable, the amplification matrix can be computed as
%\begin{gather}
%\mb{A}=\mb{A}_1^{-1}\mb{A}_2
%\end{gather}
%with
%\begin{gather}
%\mb{A}_1=\begin{bmatrix}
%2+s\Delta{}t&\cdot&-m\Delta{}t\\
%\Delta{}t\gamma&\Delta{}t\gamma{}K&M+\Delta{}t\gamma{}C\\
%\Delta{}t^2\beta&M+\Delta{}t^2\beta{}K&\Delta{}t^2\beta{}C
%\end{bmatrix},\\
%\mb{A}_2=\begin{bmatrix}
%2-s\Delta{}t&\cdot&m\Delta{}t\\
%\Delta{}t\left(\gamma-1\right)&\Delta{}t\left(\gamma-1\right)K&M+\Delta{}t\left(\gamma-1\right)C\\
%\Delta{}t^2\left(\beta-\dfrac{1}{2}\right)&M+\Delta{}t^2\left(\beta-\dfrac{1}{2}\right)K&\Delta{}tM+\Delta{}t^2\left(\beta-\dfrac{1}{2}\right)C
%\end{bmatrix}.
%\end{gather}
\section{Nonviscous Damping With Arbitrary Kernels}\label{sec:arbitrary}
\algoref{algo:single_model} alone does not enable adoption of arbitrary kernel functions, thus, it has limited applicability. To allow arbitrary kernels to be used, in this section, we present the strategy to decompose arbitrary kernels with arbitrary distributions into a series of exponential functions. Each of which can be solved by using \algoref{algo:single_model} as the basic building block.
\subsection{Sum of Exponentials}
Now consider, instead of a single exponential function, multiple exponential functions such that
\begin{gather}\label{eq:sum_exp}
g=\sum_{l=1}^{j}g_l\left(t\right)=\sum_{l=1}^{j}m_l\exp\left(-s_lt\right),
\end{gather}
where $m_l$ and $s_l$ can now be complex numbers, then
\begin{gather}\label{eq:sum_conv}
\bbf=g*\bw=\sum_{l=1}^{j}g_l*\bw=\sum_{l=1}^{j}\bbf_l.
\end{gather}
For each $\bbf_l$, \eqsref{eq:discretised_c} also holds and only involves $\bbf_l$ itself and the common quantity $\bw$. Thus,
\begin{gather}
\bbf_l=\dfrac{2-s_l\Delta{}t}{2+s_l\Delta{}t}\bbf_{l,n}+\dfrac{m_l\Delta{}t}{2+s_l\Delta{}t}\bw_n+\dfrac{m_l\Delta{}t}{2+s_l\Delta{}t}\bw.
\end{gather}

Similar to the single function case, substituting $\bbf_l$ into \eqsref{eq:eom}, differentiation yields the revised stiffness
\begin{gather}\label{eq:revised_k_multi}
\bhat{K}=\bbar{K}+\sum_{l=1}^{j}\dfrac{m_l\Delta{}t}{2+s_l\Delta{}t}\mb{T}\ddfrac{\bv}{\bu},
\end{gather}
and the revised resistance
\begin{gather}\label{eq:revised_y_multi}
\bhat{y}=\mb{y}+\sum_{l=1}^{j}\dfrac{2-s_l\Delta{}t}{2+s_l\Delta{}t}\bbf_{l,n}+\sum_{l=1}^{j}\dfrac{m_l\Delta{}t}{2+s_l\Delta{}t}\bw_n+\sum_{l=1}^{j}\dfrac{m_l\Delta{}t}{2+s_l\Delta{}t}\bw.
\end{gather}

Noting that within each sum, the operations performed are identical to that in the single function case, the complexity, in this case, is $\mathcal{O}\left(jn\right)$ for both time and space. As long as parameters $m_l$ and $s_l$ are real or pairs of complex conjugates, $\bbf$ is guaranteed to be real.
\subsection{Damping Composition}
It is possible to assign different kernels to different subsets of the velocity vector $\bv$. Formally, \eqsref{eq:sum_conv} can be further extended as
\begin{gather}
\bbf=\sum_{k=1}^{i}g^k*\bw^k,
\end{gather}
where $g^k$ is the kernel applied to $k$-th subset of $\bv$, $\bw^k=\mb{T}^k\bv$, and is expressed as the sum of exponentials,
\begin{gather}
g^k=\sum_{l=1}^{j^k}g_l^k\left(t\right)=\sum_{l=1}^{j^k}m_l^k\exp\left(-s_l^kt\right),
\end{gather}
in its explicit form,
\begin{gather}\label{eq:composition}
\bbf=\sum_{k=1}^{i}\sum_{l=1}^{j^k}\bbf_l^k=\sum_{k=1}^{i}\sum_{l=1}^{j^k}g_l^k\left(t\right)\bw^k=\sum_{k=1}^{i}\sum_{l=1}^{j^k}m_l^k\exp\left(-s_l^kt\right)\bw^k.
\end{gather}
In the above, $m_l^k$ and $s_l^k$ are the parameters for the $l$-th component of the $k$-th kernel, $\bw^k=\mb{T}^k\bv$ can be obtained by either node-based or element-based rules. For the former, it is assumed different regions (characterised by nodes) possess different damping responses. For the latter, it is assumed different elements possess different damping responses, similar to a typical assembly process, example applications can be seen elsewhere \citep{Friswell2007}.
In a broader configuration, $\mb{T}^k$ can also be defined as a combination of stiffness and mass matrices.
No matter how $\bw^k$ is constructed, the revised stiffness and resistance for each $\bbf_l^k$ only require vector--scalar operations (while $\bw^k$ itself may be computed based on $\bv$ via matrix--vector operations).

Since additivity still holds, there is no essential difference between \eqsref{eq:sum_conv} and \eqsref{eq:composition}, the same discretisation can be applied so that the revised stiffness and resistance can be obtained as
\begin{gather}\label{eq:revised_k_comp}
\bhat{K}=\bbar{K}+\sum_{k=1}^{i}\sum_{l=1}^{j^k}\dfrac{m_l^k\Delta{}t}{2+s_l^k\Delta{}t}\mb{T}^k\ddfrac{\bv}{\bu},\\
\label{eq:revised_y_comp}
\bhat{y}=\mb{y}+\sum_{k=1}^{i}\sum_{l=1}^{j^k}\dfrac{2-s_l^k\Delta{}t}{2+s_l^k\Delta{}t}\bbf_{l,n}^k+\sum_{k=1}^{i}\sum_{l=1}^{j^k}\dfrac{m_l^k\Delta{}t}{2+s_l^k\Delta{}t}\bw_n^k+\sum_{k=1}^{i}\sum_{l=1}^{j^k}\dfrac{m_l^k\Delta{}t}{2+s_l^k\Delta{}t}\bw^k.
\end{gather}
Denoting
\begin{gather}
j_\text{max}=\max_{k\in\{1,2,\cdots,i\}}\left(j^k\right),
\end{gather}
the time and space complexity is $\mathcal{O}\left(ij_\text{max}n\right)$.
\subsection{Arbitrary Kernels}
Sum of exponentials is able to provide a wide coverage of various kernel functions \citep[c.f.,][]{Adhikari2003}, as a decent amount of kernels can be equivalently expressed as sums of exponentials.

It is clear that the Fourier transform (exponential form) allows one to express arbitrary smooth function $g\left(t\right)$ as an infinite sum of exponentials, even if $g\left(t\right)$ is non-periodic. It is possible to approximate $g\left(t\right)$ if a fast converging exponential series can be found.
One can use Prony's method \citep[see, e.g.,][]{Hamming1987} and its derivations \citep{Hokanson2013} to find a proper approximation.
\citet{Du2022} adopted a similar technique, via which the convolution term is approximated by an IIR filter.
However, it tends to have a slow convergence \citep{Trudnowski1999} which inevitably leads to a large number of exponentials that would impair computational efficiency.
Alternatively, functions can be approximated by sums of exponentials or Gaussians, for further discussions on this topic, one can refer to the review by \citet{Beylkin2010}. By further adopting model reduction, \citet{Gao2022} presented a method, named as VPMR, with controllable magnitudes of exponents and fast convergence, to compute the desired approximation, formally,
\begin{gather}
\max_{t\in{}I}{\abs{g\left(t\right)-\sum_jm_j\exp\left(-s_jt\right)}}<\epsilon,
\end{gather}
where $I$ is a finite interval that could be an arbitrary subset of $\mathbb{R}^+$, $\epsilon$ is the error tolerance.

By using VPMR, it is feasible to convert nonviscous damping with arbitrary (in terms of both number and form) kernels applied to the dynamic system into the form of \eqsref{eq:composition}. Noting that $\epsilon$ is a user input, by assigning a tolerance close to (or less than) machine epsilon --- around \num{e-16} for double precision floating point representation, an accurate equivalence of arbitrary kernel function can be obtained for nonviscous damping computation. In practice, such a tolerance only needs to be smaller than the time history analysis tolerance.

\begin{breakablealgorithm}
\setstretch{1.6}
\caption{iteration body of solving nonviscously damped system with arbitrary kernels}\label{algo:vpmr}
\begin{algorithmic}
\State \textbf{*Schema: Newmark + Trapezoidal}
\State \textbf{Input}: $\bbar{K}$, $\bu$, $\bv$, $\mb{y}$, $\mb{p}$ (quantities obtained via conventional manner as if there is no nonviscous damping) and $\bbf_{l,n}^k$, $m_l^k$, $s_l^k$
\State \textbf{Output}: $\bu$
\State obtain $\bw^k$ from $\bv$ based on prescribed rules
\State \faMicrochip~compute nonviscous damping force $\bbf$\Comment{\eqsref{eq:revised_y_comp}}
\State \faMicrochip~compute revised stiffness $\bhat{K}$\Comment{\eqsref{eq:revised_k_comp}}
\State \faMicrochip~compute revised resistance $\bhat{y}=\mb{y}+\bbf$\Comment{\eqsref{eq:revised_y_comp}}
\State $\delta\bu=\bhat{K}^{-1}\left(\mb{p}-\bhat{y}\right)$
\State update and return $\bu\leftarrow\bu+\delta\bu$
\end{algorithmic}
\end{breakablealgorithm}

It is worth emphasising that converting the desired kernel functions to sums of exponentials a) is independent of the dynamic system of interest and b) is performed offline, viz., ahead of time.
Compared to the conventional solving procedure for undamped systems, the additional cost is solely governed by the number of exponentials used, as \eqsref{eq:revised_k_comp} and \eqsref{eq:revised_y_comp} only require summations of scaled matrices/vectors.
Combining with \algoref{algo:vpmr}, the following procedure can be employed to model nonviscously damped systems with arbitrary kernels.
\begin{Objective}
\begin{enumerate}
\item Determine number and form of kernels $g^k$, and the corresponding $\mb{T}^k$, to be used.
\item Determine tolerance $\epsilon$ of time history analysis.
\item Use the VPMR algorithm \citep{Gao2022} to find approximations of kernels with tolerance set to $\epsilon$ such that for each $k$,
\begin{gather}
\abs{g^k-\sum_lm_l^k\exp\left(-s_l^kt\right)}<\epsilon.
\end{gather}
\item With parameters $m_l^k$ and $s_l^k$ obtained from the VPMR algorithm, use \algoref{algo:vpmr} to solve the dynamic system.
\end{enumerate}
\end{Objective}
It shall be pointed out that in this work, the trapezoidal rule combined with the Newmark method is employed. As mentioned before, different combinations, referred here as schemas, can be used as alternatives. The specific iterative algorithms need to be derived separately. The expressions involved in \algoref{algo:vpmr}, as well as \algoref{algo:single_model}, are only applicable to the Newmark method with the trapezoidal rule.
\section{Numerical Examples}
\section{Conclusions}
In this work, a performant general--purpose algorithm integrating nonviscous damping with direct time integration methods with arbitrary kernel functions is proposed.
The proposed algorithm solves the nonviscously damped dynamic systems directly in the time domain and uses the VPMR algorithm to approximate arbitrary kernels.
The proposed algorithm can be implemented as either a global damping model or element style nonviscous dashpot, it minimises the computational cost with superior accuracy and can be integrated with various direct time integration methods.

The proposed algorithm offers an alternative tool to perform numerical analyses. However, in terms of identification and quantification of nonviscous damping, numerical results shall be validated against real structure data, and this is recommended for future research.

An open source port of the VPMR algorithm is available online\footnote{https://github.com/TLCFEM/vpmr}.
The nonviscous damping has been implemented in \texttt{suanPan} \citep{Chang2023} as both a dashpot element and a global damping model. All numerical models and the corresponding analytical solutions (if applicable) are available in this repository\footnote{https://github.com/TLCFEM/nonviscous-implementation}.
\appendix
\section{Closed Form Solution}\label{sec:analytical_sdof}
Performing Laplace transform of the following homogeneous ODE
\begin{gather}\label{eq:single_exp}
m\ddot{u}\left(t\right)+\int_0^tc\mu\exp\left(-\mu\left(t-\tau\right)\right)\dot{u}\left(\tau\right)\md{\tau}+ku\left(t\right)=0
\end{gather}
yields
\begin{gather}
m\left(s^2U-su_0-\dot{u}_0\right)
+c\mu\dfrac{sU-u_0}{s+\mu}
+kU
=0,
\end{gather}
where $U=U\left(s\right)$. Rearranging gives
\begin{gather}
\left(
s^3
+\mu{}s^2
+\dfrac{c\mu+k}{m}s
+\dfrac{k\mu}{m}\right)U
=
u_0s^2
+\left(\mu{}u_0+\dot{u}_0\right)s
+\dfrac{c\mu{}}{m}u_0
+\mu\dot{u}_0.
\end{gather}
\replace{Assuming the three roots of the characteristic equation are $r_1$, $r_2$ and $r_3$, and $U$ can be expressed as}{The three roots of the characteristic equation $s^3
+\mu{}s^2
+\dfrac{c\mu+k}{m}s
+\dfrac{k\mu}{m}$ can be solved and represented by $r_1$, $r_2$ and $r_3$, assuming $U$ can be expressed as}
\begin{gather}
U\left(s\right)=\sum_{i=1}^3\dfrac{P_i}{s-r_i},
\end{gather}
expanding and comparing,
\begin{gather}
\begin{bmatrix}
1&1&1\\
-r_2-r_3&-r_1-r_3&-r_1-r_2\\
r_2r_3&r_1r_3&r_1r_2
\end{bmatrix}
\begin{bmatrix}
P_1\\P_2\\P_3
\end{bmatrix}
=
\begin{bmatrix}
u_0\\
\mu{}u_0+\dot{u}_0\\
\dfrac{c\mu{}}{m}u_0
+\mu\dot{u}_0
\end{bmatrix}.
\end{gather}
After solving $P_i$, transforming $U$ back to the time domain gives the solution
\begin{gather}
u(t)=\sum_{i=1}^3P_i\exp\left(r_it\right).
\end{gather}
\section{VPMR Approximation of Gaussian Kernels}\label{sec:vpmr}
For kernel
\begin{gather}
g_1=1.2\sqrt{\dfrac{1}{\pi}}\exp\left(-t^2\right),
\end{gather}
the following $m_l$ and $s_l$ are used.
\begin{table}[H]
\centering\scriptsize
\caption{$m_l$ and $s_l$ of the approximation of kernel $g(t)=1.2\sqrt{1/\pi}\exp(-t^2)$}\label{tab:vpmr_g1}
\begin{tabular}{r|r|r|r}
    \toprule
                      $\Re(m_l)$ &                   $\Im(m_l)$ &                  $\Re(s_l)$ &                   $\Im(s_l)$ \\ \midrule
     \num{3.140856464484043e+01} &                              & \num{4.501691936620960e+00} &                              \\
    \num{-1.670698314099744e+01} &  \num{1.694493634189165e+01} & \num{4.495083727260961e+00} &  \num{1.039544182084329e+00} \\
    \num{-1.670698314099744e+01} & \num{-1.694493634189165e+01} & \num{4.495083727260961e+00} & \num{-1.039544182084329e+00} \\
     \num{4.311759374683535e-03} &  \num{1.018262051173052e+01} & \num{4.474910170433406e+00} & \num{-2.094934226610453e+00} \\
     \num{4.311759374683535e-03} & \num{-1.018262051173052e+01} & \num{4.474910170433406e+00} &  \num{2.094934226610453e+00} \\
     \num{1.580121702525763e+00} &  \num{1.715134770620785e+00} & \num{4.440014388559026e+00} &  \num{3.185164080597838e+00} \\
     \num{1.580121702525763e+00} & \num{-1.715134770620785e+00} & \num{4.440014388559026e+00} & \num{-3.185164080597838e+00} \\
    \num{-2.528351542418222e-01} &  \num{3.658045212954592e-02} & \num{4.387994146629901e+00} & \num{-4.337651474857473e+00} \\
    \num{-2.528351542418222e-01} & \num{-3.658045212954592e-02} & \num{4.387994146629901e+00} &  \num{4.337651474857473e+00} \\
     \num{9.686449948037756e-03} &  \num{4.313057282514074e-03} & \num{4.313868192979954e+00} & \num{-5.601534417677513e+00} \\
     \num{9.686449948037756e-03} & \num{-4.313057282514074e-03} & \num{4.313868192979954e+00} &  \num{5.601534417677513e+00} \\
    \num{-7.018890017962145e-05} &  \num{5.918104645530206e-05} & \num{4.204602737864920e+00} &  \num{7.100600444858869e+00} \\
    \num{-7.018890017962145e-05} & \num{-5.918104645530206e-05} & \num{4.204602737864920e+00} & \num{-7.100600444858869e+00} \\ \bottomrule
\end{tabular}
\end{table}
For kernel
\begin{gather}
g_2=0.4\sqrt{\dfrac{5}{\pi}}\exp\left(-5t^2\right),
\end{gather}
the following $m_l$ and $s_l$ are used.
\begin{table}[H]
\centering\tiny
\caption{$m_l$ and $s_l$ of the approximation of kernel $g(t)=0.4\sqrt{5/\pi}\exp(-5t^2)$}\label{tab:vpmr_g2}
\begin{tabular}{r|r|r|r}
    \toprule
                      $\Re(m_l)$ &                   $\Im(m_l)$ &                  $\Re(s_l)$ &                   $\Im(s_l)$ \\ \midrule
     \num{2.390615900072754e+01} &                              & \num{1.008763774835104e+01} &                              \\
    \num{-1.276225642689750e+01} &  \num{1.286047649037308e+01} & \num{1.007282883037510e+01} &  \num{2.322701258992829e+00} \\
    \num{-1.276225642689750e+01} & \num{-1.286047649037308e+01} & \num{1.007282883037510e+01} & \num{-2.322701258992829e+00} \\
     \num{5.321829762994009e-02} &  \num{7.760888995593035e+00} & \num{1.002759753347236e+01} & \num{-4.680737620510257e+00} \\
     \num{5.321829762994009e-02} & \num{-7.760888995593035e+00} & \num{1.002759753347236e+01} &  \num{4.680737620510257e+00} \\
     \num{1.194094646597810e+00} &  \num{1.321100573765225e+00} & \num{9.949275285221848e+00} &  \num{7.116484734939028e+00} \\
     \num{1.194094646597810e+00} & \num{-1.321100573765225e+00} & \num{9.949275285221848e+00} & \num{-7.116484734939028e+00} \\
    \num{-1.932626856086916e-01} &  \num{3.038377981850687e-02} & \num{9.832324567288200e+00} & \num{-9.691168168166810e+00} \\
    \num{-1.932626856086916e-01} & \num{-3.038377981850687e-02} & \num{9.832324567288200e+00} &  \num{9.691168168166810e+00} \\
     \num{7.494857745336815e-03} &  \num{3.209850541294188e-03} & \num{9.665305732814788e+00} & \num{-1.251463324996230e+01} \\
     \num{7.494857745336815e-03} & \num{-3.209850541294188e-03} & \num{9.665305732814788e+00} &  \num{1.251463324996230e+01} \\
    \num{-5.493762818993658e-05} &  \num{4.496057010788951e-05} & \num{9.418507257498408e+00} &  \num{1.586378248808019e+01} \\
    \num{-5.493762818993658e-05} & \num{-4.496057010788951e-05} & \num{9.418507257498408e+00} & \num{-1.586378248808019e+01} \\ \bottomrule
\end{tabular}
\end{table}
%Using the following parameters:
%        nc = 4.
%         n = 40.
%     order = 800.
% precision = 684.
% tolerance = 1.0000e-12.
%    kernel = 1.2*sqrt(1/pi)*exp(-t^2).


%Using the following parameters:
%        nc = 4.
%         n = 60.
%     order = 800.
% precision = 1044.
% tolerance = 1.0000e-13.
%    kernel = .4*sqrt(5/pi)*exp(-5*t^2).


\bibliography{BIB}
\end{document}