\section{Nonviscous Damping With A Single Exponential Kernel}
\subsection{Nonviscous Damped System}
Consider the equation of motion of a nonviscously damped inelastic multi-degree-of-freedom (MDOF) system,
\begin{gather}\label{eq:eom}
\mb{Y}\left(\bu,\bv,\ba\right)+\mb{F}\left(t\right)=\mb{P}\left(t\right),
\end{gather}
where $\bu=\bu\left(t\right)$, $\bv=\bv\left(t\right)=\dot{\bu}$ and $\ba=\ba\left(t\right)=\dot{\bv}$ are the displacement, velocity and acceleration vectors, $\mb{Y}=\mb{Y}\left(\bu,\bv,\ba\right)$ is the resistance vector of the system, $\mb{P}=\mb{P}\left(t\right)$ is the external load vector, and $\mb{F}$ is the nonviscous damping force which can be expressed in the form of the convolution of the kernel $f=f\left(t\right)$ and the vector $\bw$, viz. $\mb{F}\left(t\right)=f*\bw$.

Note here, $\bw$ can be either the exact velocity vector $\bv$, or the subset of $\bv$ such that they share the same size but some velocity components in $\bv$ are replaced by zeros in $\bw$ on selected DoFs. This is beneficial when it comes to compositing flexible damping that will be discussed later in this work. Formally,
\begin{gather}
\bw=\mb{T}\bv,
\end{gather}
where $\mb{T}$ is a square diagonal matrix, the diagonal entries of which are either one or zero.

Since it is an inelastic system, the stiffness matrix $\mb{K}$, the viscous damping matrix $\mb{C}$ and the mass matrix $\mb{M}$ are
\begin{gather}
\pdfrac{\mb{Y}}{\bu}=\mb{K},\qquad
\pdfrac{\mb{Y}}{\bv}=\mb{C},\qquad
\pdfrac{\mb{Y}}{\ba}=\mb{M}.
\end{gather}
The viscous damping matrix $\mb{C}$ may not be trivial as the system may consist of viscous damping components (e.g., viscous damper devices). Using $\bu$ as the basic quantity, the effective stiffness matrix $\bbar{K}$
\begin{gather}
\bbar{K}=\ddfrac{\mb{Y}}{\bu}=\mb{K}+\mb{C}\ddfrac{\bv}{\bu}+\mb{M}\ddfrac{\ba}{\bu}
\end{gather}
is the combination of the three, its specific form depends on the specific time integration method used.
\subsection{A Single Exponential Kernel}
For the moment, we focus on the scalar-valued exponential kernel function
\begin{gather}
f=f\left(t\right)=m\exp\left(-st\right),
\end{gather}
where $s$ is often denoted by the relaxation parameter $\mu$, $m$ is often denoted by $c\mu$ in which $c$ is the damping constant.
The convolution can be then expressed as
\begin{gather}\label{eq:single_conv}
\mb{F}\left(t\right)=f*\bw=\int_0^tf(t-\tau)\cdot\bw\left(\tau\right)~\md{\tau}=\int_0^tm\exp\left(-s\left(t-\tau\right)\right)\cdot\bw\left(\tau\right)~\md{\tau}.
\end{gather}
Assuming trivial initial condition $\bv\left(0\right)=\mb{0}$, \eqsref{eq:single_conv} corresponds to the solution of the following ODE,
\begin{gather}\label{eq:single_conv_ode}
\mb{F}'=-s\mb{F}+m\bw.
\end{gather}
It can be validated by solving \eqsref{eq:single_conv_ode} with the assist of the integrating factor $\exp\left(st\right)$.
\subsection{An Efficient Algorithm}
Instead of directly integrating \eqsref{eq:single_conv} using higher-order methods (such as the Runge--Kutta family), \eqsref{eq:single_conv_ode} can be combined with \eqsref{eq:eom} to develop an efficient algorithm.

In the context of a discretised iterative solving schema, \eqsref{eq:single_conv_ode} can be rewritten as follows using the backward (implicit) Euler method,
\begin{gather}\label{eq:discretised_a}
\dfrac{\mb{F}_{n+1}-\mb{F}_n}{\Delta{}t}=-s\mb{F}_{n+1}+m\bw_{n+1},
\end{gather}
in which subscripts $\left(\cdot\right)_{n+1}$ and $\left(\cdot\right)_n$ denote the corresponding quantity at $t_n$ and $t_{n+1}=t_n+\Delta{}t$.
Rearranging \eqsref{eq:discretised_a} yields
\begin{gather}\label{eq:discretised_b}
\left(1+s\Delta{}t\right)\mb{F}_{n+1}-\mb{F}_n-m\Delta{}t\bw_{n+1}.
\end{gather}

Assuming \eqsref{eq:eom} is satisfied at $t_{n+1}$\footnote{This assumption is not always valid as some time integration methods establish the EOM elsewhere, see, for example, the generalised-$\alpha$ method, the GSSSS method, the Bathe two-step method, the OALTS method, etc.}, then, accounting for both \eqsref{eq:eom} and \eqsref{eq:discretised_b}, the residual $\mb{R}$ (with the subscript $\left(\cdot\right)_{n+1}$ dropped for brevity) is
\begin{gather}
\mb{R}=\begin{bmatrix}
\mb{Q}\\\mb{W}
\end{bmatrix}=\left\{
\begin{array}{l}
\mb{Y}+\mb{F}-\mb{P},\\
\left(1+s\Delta{}t\right)\mb{F}-\mb{F}_n-m\Delta{}t\bw.
\end{array}
\right.
\end{gather}
The unknown quantity is $\bx=\begin{bmatrix}
\bu&\mb{F}
\end{bmatrix}^\mT$. Linearisation results in the following Jacobian.
\begin{gather}
\mb{J}=\begin{bmatrix}
\pdfrac{\mb{Q}}{\bu}&\pdfrac{\mb{Q}}{\mb{F}}\\[4mm]
\pdfrac{\mb{W}}{\bu}&\pdfrac{\mb{W}}{\mb{F}}
\end{bmatrix}=\begin{bmatrix}
\bbar{K}&\mb{I}\\[4mm]
-m\Delta{}t\mb{T}\ddfrac{\bv}{\bu}&\left(1+s\Delta{}t\right)\mb{I}
\end{bmatrix}.
\end{gather}
Typically, $\ddfrac{\bv}{\bu}$ reduces to a scalar constant (multiplied by an identity matrix), for example, in the Newmark method, it is $\dfrac{\gamma}{\beta\Delta{}t}$.

Noting that $\pdfrac{\mb{W}}{\mb{F}}$ is a diagonal matrix that can be easily inverted, there is no need to explicitly formulate the Jacobian. Instead, one could perform static condensation such that, from the second expression,
\begin{gather}
-m\Delta{}t\mb{T}\ddfrac{\bv}{\bu}\delta\bu+\left(1+s\Delta{}t\right)\delta\mb{F}=-\mb{W},
\end{gather}
the increment $\delta\mb{F}$ is
\begin{gather}
\delta\mb{F}=\dfrac{1}{1+s\Delta{}t}\left(-\mb{W}+m\Delta{}t\mb{T}\ddfrac{\bv}{\bu}\delta\bu\right),
\end{gather}
substituting it into the first expression yields
\begin{gather}
\bbar{K}\delta\bu+\dfrac{1}{1+s\Delta{}t}\left(-\mb{W}+m\Delta{}t\mb{T}\ddfrac{\bv}{\bu}\delta\bu\right)=-\mb{Q}.
\end{gather}
Rearranging gives
\begin{gather}
\left(\bbar{K}+\dfrac{m\Delta{}t}{1+s\Delta{}t}\mb{T}\ddfrac{\bv}{\bu}\right)\delta\bu=-\left(\mb{Q}-\dfrac{1}{1+s\Delta{}t}\mb{W}\right).
\end{gather}
By denoting
\begin{gather}
\bhat{K}=\bbar{K}+\dfrac{m\Delta{}t}{1+s\Delta{}t}\mb{T}\ddfrac{\bv}{\bu},\qquad
\bhat{Q}=\mb{Q}-\dfrac{1}{1+s\Delta{}t}\mb{W},
\end{gather}
the system to be solved is simply
\begin{gather}
\bhat{K}\delta\bu=-\bhat{Q}.
\end{gather}
Furthermore, $\bhat{Q}$ can be explicitly written as
\begin{gather}
\bhat{Q}=\mb{Y}-\mb{P}+\dfrac{1}{1+s\Delta{}t}\mb{F}_n+\dfrac{1}{1+s\Delta{}t}m\Delta{}t\bw.
\end{gather}