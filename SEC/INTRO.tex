\section{Introduction}
The equation of motion of nonviscously damped systems can be, conventionally, expressed as an integro-differential equation, namely, for elastic linear single-degree-of-freedom systems,
\begin{gather}\label{eq:single_eom}
m\ddot{u}\left(t\right)+\left(g*\dot{u}\right)\left(t\right)+ku\left(t\right)=p\left(t\right),
\end{gather}
where $u(t)$ denotes the displacement, $\dot{(\cdot)}$ denotes time derivative, and $g=g(t)$ is the kernel function. Various forms have been proposed, see a summary by \citet[][Table 1.1]{Adhikari2014}.

The convolution term in \eqsref{eq:single_eom} can be expressed in integral form such that
\begin{gather}\label{eq:conv}
\left(g*\dot{u}\right)\left(t\right)=\int_0^tg\left(t-\tau\right)\dot{u}\left(\tau\right)\md{\tau}.
\end{gather}
Approaches to compute (or approximate, depending on one's perspective) \eqsref{eq:conv} can be divided into two main categories.
\subsection{Direct Integration Methods}
Since $g(t)$ is known, \eqsref{eq:conv} can be numerically integrated such that
\begin{gather}
\int_0^Tg\left(T-\tau\right)\dot{u}\left(\tau\right)\md{\tau}\approx\sum_i^n\omega_ig\left(T-t_i\right)\dot{u}\left(t_i\right),
\end{gather}
where $t_i=\left\{t_0,t_1,t_2,\cdots,t_n\right\}\in[0,T]$ that can be either evenly or unevenly spaced, and $\omega_i$ is the corresponding integration weight, assuming $\dot{u}(t_i)$ is already obtained up to a given time $T$ at each sampling point $t_i$. If the exact $\dot{u}(t_i)$ is not available, approximations (interpolation, weighted sum, etc.) based on adjacent known values are often adopted.

Such a direct approach is intuitive, general--purpose such that it can be used for both linear and nonlinear systems with either explicit or implicit time integration methods \citep[see, e.g.,][]{Katsikadelis2019}. The simplest integration one can come up with is the rectangle rule, which is used in the implementation by \citet{Puthanpurayil2014}. Noting that the rectangle rule possesses the lowest order of accuracy possible, it effectively yields low-order overall accuracy --- even when the adopted time integration method is second-order accurate. To improve, trapezoidal rule \citep[see][]{Liu2014} and Simpson's rules \citep[see][]{Shen2019} can be used. If necessary, other high-order Newton--Cotes rules and/or more complex approximations \citep{Shen2021} can also be adopted. However, since most time integration methods (for response history analysis) are of second order, Simpson's rules would suffice.

However, no matter how the convolution term is integrated, it has to been performed for \textbf{each} substep. Given the nature of convolution, most intermediate results computed in the current substep cannot be reused in subsequent substeps. The potential exception is the values of $g(T-t_i)$ evaluated at $t_i$, which do not need to be re-evaluated if $T$ and $t_i$ satisfy certain conditions. Existing methods typically use the same set of $t_i$ in both response history analysis (to define time substeps) and numerical integration of \eqsref{eq:conv} (as the abscissae), this gives a complexity of $\mathcal{O}\left(n^2\right)$ where $n$ denotes the number of substeps. The explicit integration also requires additional kinetic assumptions, that may conflict against the ones adopted by the corresponding time integration method, to be imposed. They have significant impacts on the final analysis result \citep[see][]{Liu2014} and may even lead to poor result \cite[see][Figs. 12, 17, 25, 26]{Liu2023}.
\subsection{State Space Methods}
The state space methods essentially convert the second-order ODE into a first-order one, with which analytical methods such as modal analysis can be applied to obtain analytical solutions for linear systems. Such a conversion can be done directly in the time domain or via Laplace transform, depending on the number and form of the specific kernel function(s). By introducing additional state variables, the convolution integral can be eliminated at the cost of increasing the order of time derivatives \citep[see, e.g.,][]{Wu2019}.

The first-order ODE can also be numerically solved via discretisation.